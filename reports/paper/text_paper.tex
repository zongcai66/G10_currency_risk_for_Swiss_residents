\documentclass[a4paper, 12pt]{article}
\usepackage{parskip}
\usepackage[utf8]{inputenc} 
\usepackage{amsmath}

\title{\textbf{Which of the G10 Currencies is the Riskiest to Hold for a Swiss Resident?}}
\author{Xiao Chen 21-742-820, Yannic Laube, Bosko Todorovic, Zhi Wang}
\date{\today}

\begin{document}
\maketitle
\section{Introduction}
The rapid development of international trade and cross-border investment has promoted the foreign exchange (FX) market to be an increasingly critical component of the global economy. In international trade, the FX market not only provides the basis for currency exchange but also influences the pricing of goods, trade costs, and the competitiveness of businesses. However, uncertainties of exchange rate fluctuations have introduced significant risks and challenges to both international trade and investment.
\section{Literature Review}

\section{Methodology}

Various quantitative approaches were applied in this study to assess the exchange rate risk of the G10 currencies against the Swiss Franc. The methods used include the calculation of Value-at-Risk (VaR) through different models, the analysis of volatilities, and the investigation of the sensitivity of exchange rate returns to interest rate differentials. These approaches are described in detail in the following subsections.

\subsection{Calculation of Value-at-Risk (VaR)}

Value-at-Risk (VaR) indicates the loss that will not be exceeded with a certain probability and within a specified time period. In this analysis, a holding period of one day was chosen, and a confidence level of 95\% was used. Two different methods for calculating VaR were applied: the historical calculation and the Monte Carlo simulation for a forward-looking VaR calculation.

The historical calculation is based on the empirical distribution of historical returns. The daily exchange rate return was calculated for each currency, and the 5\% quantile of the distribution was used as the estimate for VaR. This method does not make any assumptions about the distribution of returns and thus reflects realistic market conditions.

As part of the Monte Carlo simulation for the value-at-risk (VaR) of exchange rates, 10 simulations were conducted to examine possible future scenarios for the exchange rate development of the G10 currencies against the Swiss franc (CHF). The Value-at-Risk (VaR) for the simulated price paths and simulated returns was calculated for the last period of the simulation, and the VaR was explicitly visualized in the return simulation.
First, simulated price paths for each currency were generated. These price paths were simulated using the most recent prices and daily volatility, and then visualized, without explicitly displaying the VaR.
In a subsequent step, simulated returns were calculated based on the generated price movements. These returns were visualized over the simulated periods to represent the fluctuation range of the returns. The VaR 5\% for the simulated returns was explicitly calculated for the last day of the simulation and displayed as a horizontal line in the charts. This line marks the maximum loss that will not be exceeded with 95\% probability.

\subsection{Volatility Analysis}

Volatility was calculated as a measure of the fluctuation intensity of the exchange rates. It was determined based on the standard deviation of the daily returns. High volatility values indicate increased risk, as larger fluctuations in exchange rates imply higher uncertainties. The volatilities of the G10 currencies were compared to identify which currency poses the greatest risk for a Swiss investor. Additionally, time series plots were created to visualize volatility trends during the study period.

\subsection{Regression to Analyze Interest Rate Differentials}

The sensitivity of exchange rate returns to interest rate differentials was examined using linear regression analysis. For each currency, a model was estimated where the logarithmic exchange rate return (\(\text{log\_return}_{\text{exchange}}\)) was the dependent variable, and the difference in logarithmic interest rates (\(\text{log\_diff}_{\text{interest\_rate}}\)) was the independent variable:

\[
\text{log\_return}_{\text{exchange}} = \alpha + \beta \cdot \text{log\_diff}_{\text{interest\_rate}} + \epsilon
\]

The regression analysis determined the parameters \(\alpha\) (intercept) and \(\beta\) (sensitivity of returns to interest rate differentials). Significance tests and confidence intervals were used to assess the statistical significance of the parameters. A detailed summary of the regression was created for each currency.

\section{Data}
\section{Results}
\section{Conclusion}
\section*{Appendix}
\end{document}