\documentclass{article}
\usepackage{amsmath}
\usepackage{dcolumn}
\usepackage{threeparttable}

\newcolumntype{d}[1]{D{.}{.}{#1}}

% Placeholder paragraphs with text
\usepackage{blindtext}

% No indent for new paragraphs
\setlength\parindent{0pt}

% bibliography
% \addbibresource{template-biblio.bib}


% \documentclass[a4paper, 12pt]{article}

\usepackage{parskip}
% \usepackage[utf8]{inputenc} 
% \usepackage[english]{babel}
\usepackage[letterpaper,top=2cm,bottom=2cm,left=3cm,right=3cm,marginparwidth=1.75cm]{geometry}
\usepackage{biblatex}
\addbibresource{references/refs.bib}
\usepackage{amsmath}
\usepackage{graphicx}
\usepackage{csvsimple}
\usepackage{pgfplotstable}

\title{\textbf{Which of the G10 Currencies is the Riskiest to Hold for a Swiss Resident?}}
\author{Xiao Chen 21-742-820, Yannic Laube, Bosko Todorovic, Zhi Wang}
\date{\today}

\begin{document}
\maketitle
\section{Introduction}
The rapid development of international trade and cross-border investment has promoted the foreign exchange (FX) market to be an increasingly critical component of the global economy. In international trade, the FX market not only provides the basis for currency exchange but also influences the pricing of goods, trade costs, and the competitiveness of businesses. However, uncertainties of exchange rate fluctuations have introduced significant risks and challenges to both international trade and investment~\cite{AUBOIN_RUTA_2013}~\cite{riker2020review}.

According to the Bank for International Settlements (BIS)~\cite{bis2022report}, the daily trading volume of the global FX market has exceeded \$7 trillion, with G10 currencies forming the core of this market due to their high liquidity and active trading. The G10 currencies refer to the group of ten most heavily traded and liquid currencies in the global foreign exchange market. These include the US dollar (USD), euro (EUR), Japanese yen (JPY), British pound (GBP), Swiss franc (CHF), Canadian dollar (CAD), Australian dollar (AUD), New Zealand dollar (NZD), Swedish krona (SEK), and Norwegian krone (NOK). For an open economy like Switzerland, Swiss residents holding assets denominated in other G10 currencies still face potential risks arising from exchange rate fluctuations, such as asset value depreciation, increased transaction costs, and financial market volatility affecting their investment portfolios. Moreover, G10 currencies also play a critical role in various aspects of Swiss residents' financial activities. In this context, analyzing exchange rate fluctuations and evaluating the risk characteristics of G10 currencies from perspective of a Swiss resident carries significant theoretical and practical importance.

This study aims to address the following key question: 'Which of the G10 currencies is the riskiest for Swiss residents?' By utilizing historical data and Monte Carlo simulation methods, this report evaluates the volatility, Value-at-Risk (VaR), and Expected Shortfall (ES) of each G10 currency to identify the most risky assets. The findings are expected to provide a scientific basis of investment and risk management strategies for Swiss residents.

\section{Literature Review}
A review of the existing literature provides insights into the relationship between exchange rate fluctuations, international trade, and cross-border investments. Auboin and Ruta~\cite{AUBOIN_RUTA_2013} highlight that exchange rate volatility introduces uncertainties for exporters and importers while significantly impacting cross-border investments. Companies experiencing sharp exchange rate fluctuations may experience revenue shrinkage due to foreign currency depreciation or asset devaluation.

Riker and Wickramarachi~\cite{riker2020review} further argue that persistent exchange rate changes influence multinational corporations' strategic decisions, including investment location choices and capital allocation. For instance, Avdjiev \textit{et al.}~\cite{dollar_exchange} indicate that U.S. dollar appreciation often reduces USD-denominated cross-border banking flows, thereby imposing greater financing and investment risks on countries and corporations heavily reliant on USD debt.

Researches also highlight the role of G10 currencies in the daily activities and investment strategies of Swiss residents. As a key currency in the Euro Area, the euro plays a significant role for Swiss residents in daily financial activities~\cite{engel2016exchange}~\cite{goulferni2023switzerland}. Due to Switzerland's geographical proximity to the Euro Area, EUR is commonly used alongside the Swiss franc for cross-border shopping, travel, and international payments~\cite{sif_imf_reports}. Additionally, high-liquidity G10 currencies such as the US dollar, the euro, and Japanese yen provide Swiss residents with easy access to global investment opportunities, including equities, bonds, and real estate~\cite{rogoff2000six}. Their stability helps reduce exchange rate risks~\cite{campbell2002strategic}~\cite{engel2016exchange}, ensuring more predictable investment returns and mitigating the likelihood of extreme financial losses~\cite{de1998big}. Furthermore, the liquidity and stability of G10 currencies allow Swiss residents to diversify their assets beyond the Swiss franc, a factor particularly relevant during periods of economic uncertainty~\cite{ito2020currency}. Some G10 currencies, such as the Swiss Franc and Japanese Yen, exhibit strong safe-haven properties~\cite{ranaldo2010safe}, enabling Swiss residents to preserve their wealth during periods of global financial turbulence or geopolitical instability. As a globally recognized safe-haven currency, volatility of Swiss Franc in relation to other G10 currencies is important to Swiss residents.

In addition, existing research suggests that FX market risks are often assessed through indicators such as volatility, Value-at-Risk (VaR), and Expected Shortfall (ES)~\cite{AUBOIN_RUTA_2013}~\cite{dollar_exchange}~\cite{riker2020review}. Monte Carlo simulation methods are also frequently employed to forecast future price paths, assisting investors in identifying potential extreme risks.

However, systematic comparisons of G10 currency risks from the perspective of Swiss residents are scarce in the existing literature. This report aims to address this gap by evaluating the risk characteristics of G10 currencies through historical data analysis and simulation-based methods.

\section{Methodology}

Various quantitative approaches were applied in this study to assess the exchange rate risk of the G10 currencies against the Swiss Franc. The methods used include the calculation of Value-at-Risk (VaR) through different models, the calculation of the historical Expected Shortfall (ES), the analysis of volatilities, and the investigation of the sensitivity of exchange rate returns to interest rate differentials. These approaches are described in detail in the following subsections.

\subsection{Calculation of Value-at-Risk (VaR)}

Value-at-Risk (VaR) indicates the loss that will not be exceeded with a certain probability and within a specified time period. In this analysis, a holding period of one day was chosen, and a confidence level of 95\% was used. Two different methods for calculating VaR were applied: the historical calculation and the Monte Carlo simulation for a forward-looking VaR calculation.

The historical calculation is based on the empirical distribution of historical returns. The daily exchange rate return was calculated for each currency, and the 5\% quantile of the distribution was used as the estimate for VaR. This method does not make any assumptions about the distribution of returns and thus reflects realistic market conditions.

As part of the Monte Carlo simulation for the value-at-risk (VaR) of exchange rates, 10 simulations were conducted to examine possible future scenarios for the exchange rate development of the G10 currencies against the Swiss franc (CHF). The Value-at-Risk (VaR) for the simulated price paths and simulated returns was calculated for the last period of the simulation, and the VaR was explicitly visualized in the return simulation.
First, simulated price paths for each currency were generated. These price paths were simulated using the most recent prices and daily volatility, and then visualized, without explicitly displaying the VaR.
In a subsequent step, simulated returns were calculated based on the generated price movements. These returns were visualized over the simulated periods to represent the fluctuation range of the returns. The VaR 5\% for the simulated returns was explicitly calculated for the last day of the simulation and displayed as a horizontal line in the charts. This line marks the maximum loss that will not be exceeded with 95\% probability.

\subsection{Calculation of Historical Expected Shortfall (ES)}

In addition to the calculation of Value-at-Risk (VaR), the Expected Shortfall (ES) was also computed to provide a more comprehensive measure of the tail risk. The Expected Shortfall is calculated by identifying the returns that fall below the 5\% quantile of the return distribution (as used in VaR). The average of these returns represents the Expected Shortfall. This measure is useful for understanding the potential size of losses beyond the VaR level and gives more information about the extreme risks that could affect an investor. Historical data was used to calculate the Eexoected Shortfall, and it was calculated separately for each of the G10 currencies in relation to the Swiss franc.

\subsection{Volatility Analysis}

Volatility was calculated as a measure of the fluctuation intensity of the exchange rates. It was determined based on the standard deviation of the daily returns. High volatility values indicate increased risk, as larger fluctuations in exchange rates imply higher uncertainties. The volatilities of the G10 currencies were compared to identify which currency poses the greatest risk for a Swiss investor. Additionally, time series plots were created to visualize volatility trends during the study period.

\subsection{Regression to Analyze Interest Rate Differentials}

The sensitivity of exchange rate returns to interest rate differentials was examined using linear regression analysis. For each currency, a model was estimated where the logarithmic exchange rate return (\(\text{log\_return}_{\text{exchange}}\)) was the dependent variable, and the difference in logarithmic interest rates (\(\text{log\_diff}_{\text{interest\_rate}}\)) was the independent variable:

\[
\text{log\_return}_{\text{exchange}} = \alpha + \beta \cdot \text{log\_diff}_{\text{interest\_rate}} + \epsilon
\]

The regression analysis determined the parameters \(\alpha\) (intercept) and \(\beta\) (sensitivity of returns to interest rate differentials). Significance tests and confidence intervals were used to assess the statistical significance of the parameters. A detailed summary of the regression was created for each currency.

\section{Data}
The compiled datasets contain overnight interest rates and foreign exchange rate data from multiple sources.

\subsection{Overnight Interest Rates}
Overnight interest rate data for Norway, Japan, the United Kingdom, the United States, Germany, Australia, New Zealand, and Canada were obtained from the Federal Reserve Economic Data (FRED) database. These data represent monthly average rates, transformed and published according to FRED's standards. Data for Switzerland was retrieved from the Swiss National Bank's (SNB) official API, specifically using the SARON (Swiss Average Rate Overnight) series to reflect overnight interest rates. Similarly, overnight interest rate data for Sweden was sourced from the Swedish Riksbank’s Interest Rates and Exchange Rates Statistics, where the repo effective rate is presented in monthly values. The dataset spans the period from January 2000 to October 2024.

\subsection*{Foreign Exchange Rates}
Daily foreign exchange rate data for G10 currencies against the Swiss Franc (CHF) were collected using the YFinance library. The dataset includes exchange rates for the Euro (EUR/CHF), British Pound (GBP/CHF), US Dollar (USD/CHF), Canadian Dollar (CAD/CHF), Swedish Krona (SEK/CHF), Japanese Yen (JPY/CHF), Australian Dollar (AUD/CHF), New Zealand Dollar (NZD/CHF), and Norwegian Krone (NOK/CHF). The data spans the period from January 1, 2004, to October 21, 2024.

\section{Results}
\subsection{Basic Risk Measures}

\subsection{VaR Calculation}
In the historical VaR analysis from Figure~\ref{fig:historical_VaR} and Table~\ref{tab:var}, AUDCHF, CADCHF, NOKCHF, and NZDCHF all showed losses exceeding 1\%. 

\begin{table}[H]
\centering
\caption{Historical VaR (5\%) and Monte Carlo VaR (5\%) for each currency pair.} 
\label{tab:var}
\pgfplotstabletypeset[
    col sep=comma,
    fixed zerofill,
    columns/Currency Pair/.style={string type},
    columns/Historical VaR/.style={
        fixed,
        precision=6,
        zerofill
    },
    columns/Monte Carlo VaR/.style={
        fixed,
        precision=6,
        zerofill
    },
    every head row/.style={before row=\hline, after row=\hline},
    every last row/.style={after row=\hline},
]{reports/figures/VaR_results.csv}
\end{table}

\begin{figure}[H]
    \centering   \includegraphics[width=0.75\linewidth]{reports/figures/var_5_percent_comparison_plot.png}
    \caption{Historical VaR of G10 currencies.}
    \label{fig:historical_VaR}
\end{figure}

However, in the Monte Carlo simulation, only AUD and NZD still indicated a VaR loss of over 1\%, suggesting that AUD and NZD exhibit high risk persistence as can be shown in Figure~\ref{fig:monte_carlo_var_simulation_AUDCHF_vs_CHF} and Figure~\ref{fig:monte_carlo_var_simulation_NZDCHF_vs_CHF}.

\begin{figure}[H]
    \centering   
    \includegraphics[width=0.75\linewidth]{reports/figures/monte_carlo_var_simulation_AUDCHF_vs_CHF.png}
    \caption{Monte Carlo VaR Simulation of AUDCHF vs. CHF}  \label{fig:monte_carlo_var_simulation_AUDCHF_vs_CHF}
\end{figure}

\begin{figure}[h]
    \centering   \includegraphics[width=0.75\linewidth]{reports/figures/monte_carlo_var_simulation_NZDCHF_vs_CHF.png}
    \caption{Monte Carlo VaR Simulation of NZDCHF vs. CHF}  \label{fig:monte_carlo_var_simulation_NZDCHF_vs_CHF}
\end{figure}

In other words, both historical data and simulations indicate that these two currencies carry sustained risk characteristics, and investors should be particularly cautious when considering these currency pairs. It is worth noting that SEK showed increased risk in the Monte Carlo simulation, with its VaR exceeding 1\%, transitioning from a low-risk to a high-risk profile. This suggests that SEK may face greater uncertainty and potential risks in future market volatility.

\subsection{Price Simulation}
Regarding price simulations, GBP shows the most significant price volatility, approximately 0.4 (Figure~\ref{fig:monte_carlo_price_simulation_GBPCHF_vs_CHF}), indicating high future price uncertainty for GBP pairs. 

\begin{figure}[h]
    \centering  \includegraphics[width=0.75\linewidth]{reports/figures/monte_carlo_price_simulation_GBPCHF_vs_CHF.png}
    \caption{Monte Carlo Price Simulation of GBPCHF vs. CHF}  \label{fig:monte_carlo_price_simulation_GBPCHF_vs_CHF}
\end{figure}

Volatility of AUDCHF is around 0.33 (Figure~\ref{fig:monte_carlo_price_simulation_AUDCHF_vs_CHF}), and of NZDCHF is approximately 0.27 (Figure~\ref{fig:monte_carlo_price_simulation_NZDCHF_vs_CHF}), also reflecting considerable price uncertainty. 

\begin{figure}[h]
    \centering \includegraphics[width=0.75\linewidth]{reports/figures/monte_carlo_price_simulation_AUDCHF_vs_CHF.png}
    \caption{Monte Carlo Price Simulation of AUDCHF vs. CHF} \label{fig:monte_carlo_price_simulation_AUDCHF_vs_CHF}
\end{figure}

\begin{figure}[h]
    \centering  \includegraphics[width=0.75\linewidth]{reports/figures/monte_carlo_price_simulation_NZDCHF_vs_CHF.png}
    \caption{Monte Carlo Price Simulation of NZDCHF vs. CHF} \label{fig:monte_carlo_price_simulation_NZDCHF_vs_CHF}
\end{figure}
This suggests that these three currencies have a wide potential price fluctuation range, with correspondingly high investment risks and potential returns.

To summarize the results from both VaR and price simulations, AUD and NZD exhibit both high risk persistence and significant price volatility, indicating that they are high-risk and high-reward currency pairs suitable only for investors willing to take substantial risks. GBP, despite having a relatively low VaR, has significant price volatility of around 0.4, indicating strong future price fluctuations. Investing in this currency pair requires fully considering the risks associated with its price movements. Therefore, investors should make informed choices regarding these currency pairs based on their own risk tolerance and assessment of future market conditions.

Figure~\ref{fig:monte_carlo_var_simulation_JPYCHF_vs_CHF} and Figure~\ref{fig:monte_carlo_price_simulation_JPYCHF_vs_CHF} highlight that JPY has the lowest price volatility, around 0.0022, indicating extremely low future price fluctuations. Combined with its historical VaR and Monte Carlo simulation results consistently remaining around 0.9\%, JPY demonstrates very stable risk characteristics. This suggests that JPY is relatively stable in the current market environment. Its low volatility and steady VaR make it an excellent hedging option in an investment portfolio, especially suitable for investors seeking lower-risk investments.

\begin{figure}[h]
    \centering  \includegraphics[width=0.75\linewidth]{reports/figures/monte_carlo_var_simulation_JPYCHF_vs_CHF.png}
    \caption{Monte Carlo VaR Simulation of JPYCHF vs. CHF}   \label{fig:monte_carlo_var_simulation_JPYCHF_vs_CHF}
\end{figure}

\begin{figure}[h]
    \centering
    \includegraphics[width=0.75\linewidth]{reports/figures/monte_carlo_price_simulation_JPYCHF_vs_CHF.png}
    \caption{Monte Carlo Price Simulation of JPYCHF vs. CHF}  \label{fig:monte_carlo_price_simulation_JPYCHF_vs_CHF}
\end{figure}

\subsection{Interest Rate Regression Analysis}
The regression results for JPY show a significant negative sensitivity to interest rate changes in Table~\ref{tab:regression} ($\beta$ = -3.25, highly significant), indicating that the return rate of the yen fluctuates significantly with changes in interest rates. However, due to the stability of the Japanese economy and the market confidence in the yen's role as a safe-haven currency, the overall volatility of the yen is controlled. This explains why the VaR and price volatility of yen are both low, meaning that in the actual market, the price volatility of the yen is controlled by certain factors that may reduce its extreme short-term risk.

\begin{table}[h]
\centering
\caption{Regression summaries of exchange rate returns on interest rate differentials.} 
\label{tab:regression}
\resizebox{\textwidth}{!}{
\pgfplotstabletypeset[
    col sep=comma,
    string type,
    columns={Country, Alpha, Beta, Alpha Std.Err., Beta Std.Err., Alpha t, Beta t, Alpha P>|t|, Beta P>|t|},
    columns/Country/.style={string type, column name=Country},
    columns/Alpha/.style={fixed, precision=4, column name=$\alpha$},
    columns/Beta/.style={fixed, precision=4, column name=$\beta$},
    columns/Alpha Std.Err./.style={fixed, precision=4, column name=std.err\_$\alpha$},
    columns/Beta Std.Err./.style={fixed, precision=4, column name=std.err\_$\beta$},
    columns/Alpha t/.style={fixed, precision=4, column name=t-value\_$\alpha$},
    columns/Beta t/.style={fixed, precision=4, column name=t-value\_$\beta$},
    columns/Alpha P>|t|/.style={fixed, precision=4, column name=p-value\_$\alpha$},
    columns/Beta P>|t|/.style={fixed, precision=4, column name=p-value\_$\beta$},
    every head row/.style={before row=\hline, after row=\hline},
    every last row/.style={after row=\hline}
]{reports/figures/regression_summaries.csv}
}
\end{table}
In contrast , despite the regression results for NZD showing a low sensitivity to interest rate changes ($\beta$ = 0.13, and not significant), which implies that interest rate changes do not have a major impact on NZD in the regression model. However, VaR and price volatility of NZD are high. This indicates that NZD has considerable volatility in the actual market, then there may be other factors significantly affect NZD, such as commodity price fluctuations (as New Zealand is a major commodity exporter~\cite{blundell1990exchange}) and speculative behavior in the foreign exchange market. These factors may not have been adequately captured in the interest rate regression model.

In conclusion, ...

\section{Conclusion}
jbjvboiigotulh;joiouti
\section*{Appendix}
jfogiougiyctugh;oii
\begin{figure}[h]
    \centering  \includegraphics[width=0.48\linewidth]{reports/figures/monte_carlo_price_simulation_CADCHF_vs_CHF.png} \label{fig:monte_carlo_price_simulation_CADCHF_vs_CHF}
    \includegraphics[width=0.48\linewidth]{reports/figures/monte_carlo_var_simulation_CADCHF_vs_CHF.png} \label{fig:monte_carlo_var_simulation_CADCHF_vs_CHF}
    \caption{\footnotesize Monte Carlo price siulation (left) and VaR simulation (right) for CAD-CHF.}
\end{figure}
\printbibliography
\end{document}